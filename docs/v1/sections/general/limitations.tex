\section{Limitations}

\subsection{Résilience}

Le problème de la résilience se pose au niveau de l'API Mazerunner et de l'application web. En effet, en l'état, ces services sont situés sur une seule machine et par conséquent une panne sur cette machine entraine l'indisponibilité de l'ensemble du système.
Le même problème se pose pour les load-balancers ou le HAproxy mis en place pour pallier aux potentiels crash des machines. 
En l'état, nous avons installé un voire deux proxys pour chacun des composants, hors si ces proxys tombent alors le composant est inaccessible. 
La solution évidente serait donc d'augmenter le nombre de machines dédiées pour améliorer la résilience. Mais se pose dès lors la question de savoir à quel moment on considère la robustesse du système suffisante. 
Dans notre cas, nous avons décidé d'accepter ces points de faiblesse mais dans un environnement de production, nous recommandons une plus grand exigence en terme de redondance.

\subsection{Scala}

