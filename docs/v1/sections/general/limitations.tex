\section{Limitations}

\subsection{Résilience}

Le problème de la résilience se pose au niveau de l'API Mazerunner et de l'application web. En effet, en l'état, ces services sont situés sur une seule machine et par conséquent une panne sur cette machine entraine l'indisponibilité de l'ensemble du système. Le même problème se pose pour les load-balancers ou le HAproxy mis en place pour pallier aux potentiels crash des machines. En l'état nous avons installé un voire deux proxys pour chaque composant, hors si ces proxys tombent alors le composant est inaccessible. La solution évidente serait donc d'augmenter le nombre de machines dédiées pour améliorer la résilience. Mais se pose dès lors la question de savoir à quel moment on considère la robustesse du système suffisante. Dans notre cas, nous avons décidé d'accepter ces points de faiblesse mais dans un environnement de production, nous recommandons une plus grand exigence en terme de redondance.

\subsection{Déploiements}

Le déploiement se base sur une série de scripts Python.
Or ces scripts sont exécutés de manière séquentielle, et bien que le temps de déploiement reste raisonnable il est possible de l'améliorer en exécutant le déploiement en parallèle pour chaque machine.
Etant donné que le déploiement actuel fonctionne nous avons préféré ne pas apporter de modifications si importantes, mais cela reste une piste d'amélioration possible avec des frameworks comme Fabric.
