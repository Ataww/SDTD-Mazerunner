\section{Neo4j}

\subsection{Présentation générale}

Neo4j est un système de gestion de base de données orienté graphes. Le code source, écrit en Java, est développé par Neo Technology.

\subsection{Caractéristiques}

La base de données de graphes offrent de meilleures performances dans le traitement de requêtes utilisant des relations entre objets. Au lieu d'utiliser des jointures comme dans le bases de données relationnelles, Neo4j utilise des outils de parcours de graphes. Ces outils permettent également de faciliter des cas d'usages exploitant au maximum les relations.

Les données ne sont pas stockées de manière structurée dans Neo4j. Cela porte l'avantage d'adapter la base à la modification continue des données. En outre, le temps de développement de la base et le coût de sa maintenance s'en trouvent réduits.

Neo4j utilise le langage Cypher pour la description des requêtes. 

\subsection{Cas d'utilisation privilégiés}

\begin{itemize}
      \item Gestion de réseau
      \item Réseaux sociaux
      \item Recommandation
      \item Géo-spatial
\end{itemize}

\subsection{Installation}

Neo4j requiert Java 7.

\begin{lstlisting}
wget -O - http://debian.neo4j.org/neotechnology.gpg.key | apt-key add -
echo 'deb http://debian.neo4j.org/repo stable/' > /etc/apt/sources.list.d/neo4j.list
apt-get update
apt-get install neo4j
\end{lstlisting}

Pour démarrer le serveur :
\begin{lstlisting}
/etc/init.d/neo4j-service start
Après extraction du dossier, 
\end{lstlisting}

\subsection{Clusterisation}

Pour que neo4j fonctionne en cluster, trois rôles doivent être mis en place via les fichiers de configuration :

\begin{itemize}
      \item Master : c'est le point d'entrée principal en écriture. Lorsqu'un changement est effectué sur la base, il réplique l'information vers les esclaves.
      \item Slave : il assure la pérénité de l'information si le master venait à tomber. On y accède également en lecture.
      \item Arbitre : il échange régulièrement des heartbeats avec l'ensemble des serveurs pour déterminer l'état du cluster. Il pilote l'élection du master lorsque c'est nécesaire.
\end{itemize}

Pour MazeRunner, on a un noeud master, neo4j-1, un noeud esclave, neo4j-2 et enfin un noeud arbitre, neo4j-3.

L'équilibrage de charge est réalisé en mode proxy à l'aide de HAproxy. Il fait l'interface avec les requêtes HTTP et Bolt et les redirige vers les serveurs master ou slave selon leur état d'occupation. Ainsi, tous les composants ayant à intéragir avec la base de données dirigent leurs requêtes vers neo4j-3, qui contient la configuration HAproxy.

\subsection{Bolt}

Bolt est utilisé à la place de HTTP comme protocole réseau. Ce nouveau protocole a été créé spécifiquement pour Neo4j de manière à optimiser les performances. Ce protocole est utilisé par l'API MazeRunner.

\subsection{Import de données}

Neo4j fournit un outil très utile pour l'import automatique de données de fichiers .csv : neo4j-import. Il faut combiner les options de la commande neo4j-import avec les bons headers dans les fichiers .csv :

\begin{itemize}
      \item --into : pour spécifier l'emplacement de la base à créer
      \item --nodes : pour créer de nouveaux noeuds. Dans le fichier .csv, on doit identifier une colonne d'ID en ajoutant le suffixe ':ID' au nom de la colonne, et on donne le nom de la classe du noeud dans une colonne portant le header ':LABEL'
      \item --relationships : pour créer des relations. Trois colonnes sont nécessaires : la première porte le header ':START_ID' et la deuxième ':END_ID' pour signifier l'origine et le point d'arrivée de la relation. La troisième colonne doit être intitulée ':TYPE' et indiquer le nom de la relation.
\end{itemize}

L'import de la base dans ce projet est réalisé dans le script neo4j_install, dans la fonction config_neo4j.
